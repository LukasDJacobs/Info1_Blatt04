\section{Heimübung 1}
Erstellen Sie ein F\# Programm, welches...
\subsection{a)\dots vier beliebige Listen anlegt und mit Werten füllt und diese auf der Konsole ausgibt.}
\begin{enumerate}
	\item Füllen Sie die Liste1 mit Literalen.
	\item Füllen Sie die Liste2 mit Hilfe des cons (::) Operator.
	\item Füllen Sie die Liste3 mit Hilfe des range Konstrukts.
	\item Erstellen Sie eine neue Liste, indem Sie zwei bestehenden Listen verknüpfen.
\end{enumerate}

\subsection{b)\dots die folgenden Basisoperatoren von F\# auf allen Listen anwendet und diese auf der Konsole ausgibt.}
\begin{enumerate}
	\item Geben Sie jeweils \textit{Head} und \textit{Tail} der Listen auf der Konsole aus.
	\item Berechnen Sie die Summe der Listenelement mit der Funktion \textit{Sum}.
	\item Berechnen Sie den Durchschnitt aller Listenelemente mit der Funktion	\textit{Average}.
\end{enumerate}

\subsection{c)\dots folgende Operation implementiert und diese auf der Konsole ausgibt.}
\begin{enumerate}
	\item Berechnen Sie die Summe aller Elemente einer Liste rekursiv. Hinweis: Nutzen Sie hierfür \textit{Head} und \textit{Tail}.
	\item Berechnen Sie die Summe aller Elemente einer Liste iterativ.
\end{enumerate}

\paragraph*{}\textit{Aufgaben a) bis c) werden separat als .fs-Dateien abgegeben.}
\paragraph*{Hinweis:}Achten Sie darauf Ihren Code zu kommentieren!

\subsection{d) Beschreiben Sie Ihr Vorgehen, um den Durchschnitt einer Liste mit Integer Variablen zu berechnen. Warum müssen Sie sich diese Gedanken machen, wenn es doch die Funktion \textit{Average} bereits gibt?} 